This dissertation investigates participatory decision‑making in online communities through the
design, implementation and evaluation of a decision‑making tool deployed within an online chat platform to support governance processes.  Computational social
choice theory offers mechanisms to aggregate individual preferences into collective decisions,
yet practical deployments must address real‑world constraints.  Drawing inspiration from
value‑sensitive design and recent work on self‑organising socio‑technical systems, we seek to
empower community members to propose, debate and vote on policies in a transparent and
accountable manner.  The contributions of this thesis are fourfold: (i)~a critical literature
review of social choice theory, participatory design and algorithmic governance; (ii)~the
architecture and implementation of a software component supporting multiple voting mechanisms, flexible weighting schemes and an extensible data model; (iii)~an empirical evaluation comparing a standard plurality vote with a weighted voting mechanism; and (iv)~a discussion of limitations, sustainability considerations and future directions, including a machine‑learning–based ``what‑if'' predictor to assist in choosing voting mechanisms.  The evaluation indicates that the weighted token‑allocation mechanism enhanced participants' perceptions of fairness and satisfaction compared to a standard plurality vote.
\end{abstract}
