\doublespacing % Do not change - required

\chapter{Supplementary Materials}
\label{appendix}

\thesisspacing % Do not change - required

This appendix provides supplementary details to support the main text.

\section{Pre‑Voting Survey}
\label{appendix:pre-survey}

The pre‑voting survey gathers baseline information about participants’
experience with Discord, their familiarity with voting methods and their
attitudes towards fairness, speed and trust in decision procedures.  Unless
otherwise indicated, items are rated on a five‑point scale where higher
values indicate stronger agreement.

\begin{enumerate}
    \item \textbf{Consent to participate}.  Participants indicate whether they
    consent to participate in the study (Yes/No).
    \item \textbf{Name}.  Participants provide their name to match pre‑ and
    post‑responses.  Names will be anonymised during analysis.
    \item \textbf{Discord usage}.  ``Have you used Discord before?'' (Yes/No).
    \item \textbf{Group discussion frequency}.  ``How often do you
    participate in group discussions?'' (Always, Often, Sometimes, Rarely,
    Never).
    \item \textbf{Knowledge of plurality voting}.  ``Have you heard of
    plurality voting?'' (Yes/No).
    \item \textbf{Knowledge of weighted voting}.  ``Have you heard of
    weighted voting?'' (Yes/No).
    \item \textbf{Fairness importance}.  ``I care that the decision feels
    fair to the group.'' (1–5).
    \item \textbf{Speed vs fairness}.  ``I care more about speed than
    perfect fairness.'' (1–5).
    \item \textbf{Comfort with Discord polls}.  ``I’m comfortable using
    Discord polls for decisions.'' (1–5).
    \item \textbf{Openness to new methods}.  ``I’m open to trying different
    voting methods.'' (1–5).
    \item \textbf{Trust in explanation}.  ``I trust results more when the
    method is explained clearly.'' (Strongly disagree, Disagree, Neutral,
    Agree, Strongly agree).
    \item \textbf{Discord server confirmation}.  Participants confirm that
    they have joined the study’s Discord server (Yes).
\end{enumerate}

\section{Post‑Voting Survey}
\label{appendix:post-survey}

The post‑voting survey is administered after participants complete both
voting sessions.  It measures changes in understanding, satisfaction with
outcomes and perceptions of fairness and legitimacy for each method, as
well as experiences with token allocation and preferences for future
voting methods.

\subsection*{Change in understanding}
\begin{itemize}
    \item ``Compared to before, my understanding of voting methods is
    now …'' with response options: Much better, Better, Same, Worse,
    Much worse.
\end{itemize}

\subsection*{Outcome satisfaction for normal voting}
Participants rate each aspect of the final plan from 1 (Very poor) to
5 (Excellent) for the outcome produced by the normal voting session:
\begin{itemize}
    \item Breakfast.
    \item Transport.
    \item Hangout location.
    \item Lunch.
    \item Evening activity.
    \item Dinner.
\end{itemize}

\subsection*{Outcome satisfaction for weighted voting}
Participants provide analogous ratings for the outcome produced by the
weighted voting session:
\begin{itemize}
    \item Breakfast.
    \item Transport.
    \item Hangout location.
    \item Lunch.
    \item Evening activity.
    \item Dinner.
\end{itemize}

\subsection*{Outcome fairness and legitimacy for normal voting}
On a five‑point scale from Strongly disagree to Strongly agree, participants
indicate the extent to which:
\begin{itemize}
    \item The final decision felt fair.
    \item The process reflected group preferences.
    \item I would reuse this process in my groups.
    \item I feel satisfied with the outcome.
\end{itemize}

\subsection*{Outcome fairness and legitimacy for weighted voting}
Using the same scale as above, participants evaluate the weighted voting
outcome with respect to fairness, reflection of preferences, willingness to
reuse the process and satisfaction with the outcome.

\subsection*{Token allocation experience}
Participants respond to the following statements using a five‑point scale
from Strongly disagree to Strongly agree:
\begin{itemize}
    \item ``Thinking about the process of allocating tokens made me
    consider my choices more carefully.''
    \item ``I feel that I was able to use my tokens effectively.''
    \item ``The weighted voting process made me feel my voice was heard
    more clearly than in the normal vote.''
\end{itemize}
Participants are also asked whether their top choice on the main aspect
(e.g., the primary activity) won in each session (Yes, No, Close
second).

\subsection*{Preference for method}
Participants indicate which method they would prefer to use in the future
for a similar group decision, selecting one of the following options:
\begin{itemize}
    \item I would prefer normal voting.
    \item I would prefer weighted voting.
    \item I would prefer weighted voting, but only if I get more
    comfortable with it.
    \item I have no preference.
\end{itemize}

\subsection*{Discord poll usability}
Participants report whether the polls were easy to understand on Discord
(Yes, No, Could be improved).

\subsection*{Open feedback}
Participants are invited to provide one suggestion or comment to improve
the process in future iterations.  Responses to this open‑ended item will
be analysed qualitatively.

\section{Token Allocation Procedure}
\label{appendix:token}

This appendix describes the procedure followed during weighted voting
sessions.  Its purpose is to ensure that all participants understand
how to invest their tokens and to standardise data collection.

\begin{enumerate}
    \item \textbf{Token endowment}.  At the beginning of a weighted
    session each participant receives a fixed budget of 14 tokens.  The
    budget appears as a progress bar at the top of the voting view.
    \item \textbf{Allocation interface}.  For each alternative a pair
    of buttons labelled “+” and “–” allows the participant to increment
    or decrement the number of tokens invested in that alternative.
    Tokens are integer values; fractional tokens are not permitted.
    \item \textbf{Budget constraint}.  The sum of tokens allocated across
    all alternatives cannot exceed the initial budget.  The progress bar
    updates dynamically to show how many tokens remain.  If a participant
    attempts to over‑invest, the interface displays an error message and
    prevents the action.
    \item \textbf{Review and confirmation}.  Participants may adjust
    allocations until satisfied.  Once the “Confirm” button is
    selected, a summary of the allocations is displayed.  Participants
    must explicitly confirm this summary; after confirmation, the
    allocations are recorded in the database and cannot be changed.
    \item \textbf{Multiple decisions}.  In campaign mode a single token
    budget applies across several proposals (e.g., breakfast, transport,
    activity).  Participants decide how to distribute their tokens
    across these decisions.  Remaining tokens can be carried forward to
    subsequent decisions within the same campaign but cannot be carried
    over to future campaigns.
    \item \textbf{Meaning of tokens}.  Tokens represent the strength of
    preference rather than monetary value.  Participants are instructed
    that investing more tokens in an alternative indicates a stronger
    desire to see that option selected.  Unused tokens have no value
    and do not confer any advantage in future votes.
\end{enumerate}

Participants are reminded of these rules before beginning the weighted
session.  The aim is to encourage thoughtful consideration of how to
allocate limited influence across multiple alternatives and to avoid
strategic behaviours based on misconceptions about the token system.
