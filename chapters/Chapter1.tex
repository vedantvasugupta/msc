\doublespacing % Do not change - required

\chapter{Introduction}
\label{ch:introduction}

%%%%%%%%%%%%%%%%%%%%%%%%%%%%%%%%%%%%%%%
% IMPORTANT
\begin{spacing}{1} %THESE FOUR
\minitoc % LINES MUST APPEAR IN
\end{spacing} % EVERY
\thesisspacing % CHAPTER
% COPY THEM IN ANY NEW CHAPTER
%%%%%%%%%%%%%%%%%%%%%%%%%%%%%%%%%%%%%%%

\section{Background and Motivation}

Collective decision‐making is a fundamental activity of human societies.  When
groups must choose a policy, allocate resources or establish a social norm, they
need to aggregate diverse preferences into a single outcome.  The field of
\emph{social choice theory} formalises this problem and studies the properties of
different aggregation mechanisms.  Arrow’s impossibility theorem famously
showed that no rank‑ordering rule can simultaneously satisfy four key axioms—unrestricted domain, Pareto
efficiency, independence of irrelevant alternatives, and non‑dictatorship \cite{Arrow1951}.  Subsequent theorists have
argued that the real problem of social choice is to determine which
combination of gains and losses across individuals is socially
preferable \cite{Black1958, Okasha2011}.  These results highlight fundamental
trade‑offs between fairness, representativeness and manipulability.

With the proliferation of online communities on platforms such as Discord and
Slack, collective decision‑making increasingly occurs through digital
interfaces.  However, the native polling tools offered by these platforms are
rudimentary: they typically limit the number of options, do not allow ranked or
weighted voting, and provide little transparency about how winners are
determined.  Communities therefore require governance tools that embed richer
voting mechanisms while remaining accessible to non‑experts.

Recent work in socio‑technical systems argues that legitimate governance
requires continuous civic participation across the \emph{engage–envision–enact}
cycle.  Mertzani \emph{et~al.} \cite{Mertzani2023Engage} note that democracy should not be treated as a
static, universally optimal regime; anthropological and political evidence
demonstrates that democratic institutions evolve and may require adaptation to
remain fit for purpose.  They argue that legitimate
consent arises when governance processes are meaningful, informed and
revocable, and that communities must be empowered to reflect on and
reconfigure their social arrangements.  Related research on socially‑guided
machine learning emphasises that hybrid human–AI systems risk reinforcing
existing power asymmetries unless those affected participate in designing and
modifying decision mechanisms \cite{Mertzani2025SGML}.  Together, these
perspectives motivate the design of participatory governance tools that
encourage reflection, support multiple voting methods and respect user
autonomy.

\paragraph{Limitations of native polls.}  The standard polling
widgets provided by Discord and similar platforms present only a small
 number of options and permit users to select, at most, one.  They
implement a simple plurality rule whereby the alternative with the
largest number of votes wins; no majority is required, and ties are
handled arbitrarily \cite{Black1958}.  Such rules ignore the intensity of
preferences and cannot accommodate ranked or weighted ballots.  In
 addition, the User Interface (UI) does not allow participants to express
indifference or approval for multiple options.  From a theoretical
 perspective, these limitations map directly to the impossibility
results of social choice: Arrow’s theorem shows that no ranked‑choice
rule can simultaneously satisfy fairness criteria such as the
independence of irrelevant alternatives and non‑dictatorship; all
rules entail trade‑offs \cite{Arrow1951}.  By restricting
choice to a single checkbox, native polls implicitly select one point
in this trade‑off space without giving communities the ability to
choose alternative mechanisms.  Our motivation is therefore to
construct a platform that exposes multiple voting rules, supports
the expression of preferences through ranking and weighting, and remains usable for
non‑experts despite the constraints of the Discord interface.

\section{Objectives}

The overarching research question of this thesis is:

\begin{quote}
\emph{How can an online governance tool support diverse and expressive voting mechanisms while remaining usable and trustworthy to community members?}
\end{quote}

To address this question, we pursue four objectives:
\begin{enumerate}
    \item \textbf{Theoretical review}: Survey the foundations of social choice
    theory, PD and algorithmic governance, emphasising the
    trade‑offs uncovered in the literature.
    \item \textbf{System design}: Formally specify and implement a
    governance tool integrated within the chat platform that enables proposal submission and
    supports multiple voting mechanisms—plurality, Borda, approval,
    instant runoff and Condorcet—together with an optional weighted
    scheme.  The design emphasises a clear architecture with
    well‑defined components, data models and algorithms rather than a
    chronological narrative of software development.  All interface
    elements must respect the platform’s limitations on modals and
    components while remaining intuitive for non‑expert users.
    \item \textbf{Empirical evaluation}: Design and conduct a controlled
    experiment using a \emph{Day~Out} scenario.  Ten participants took
    part in both plurality and weighted voting sessions and completed pre‑ and
    post‑surveys measuring their satisfaction with the process and outcome, perceptions of fairness and the extent to which the final decision reflected the group’s preferences, as well as their willingness to reuse the mechanism.  The resulting data are analysed in Chapter~\ref{ch:experiments}, where we report descriptive statistics, compare the two voting methods and summarise participants’ preferences and qualitative feedback.
    \item \textbf{Reflection and impact}: Reflect on the strengths and
    limitations of the system, propose future improvements—including a
    machine‑learning–based \emph{what‑if} predictor—and assess the
    environmental and societal impacts of participatory governance.
\end{enumerate}

\section{Structure of the Thesis}

The remainder of the thesis is organised as follows.  Chapter~\ref{ch:litreview}
presents a literature review of social choice theory, participatory and
power‑sensitive design, digital voting systems and fairness criteria.
Chapter~\ref{ch:method} introduces the methodology and high‑level system
design, presenting the guiding principles and architecture of the governance
tool.  Chapter~\ref{ch:implementation} details the technical implementation,
describing the components, data model, algorithms and user‑interface flows.
Chapter~\ref{ch:experiments} describes the experimental setup, reports the collected data and compares the plurality and weighted voting conditions.
Chapter~\ref{ch:discussion} interprets the results, situates them within
the broader literature and critiques the system’s strengths and limitations.
Chapter~\ref{chLast} contains reflections on legal, ethical, environmental
and inclusion aspects of the work, together with project limitations and
technical challenges.  Finally, Chapter~\ref{ch:conclusions} concludes the
thesis and outlines directions for future research.