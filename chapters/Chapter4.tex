\doublespacing % Do not change - required

\chapter{System Implementation}
\label{ch:implementation}

%%%%%%%%%%%%%%%%%%%%%%%%%%%%%%%%%%%%%%%
% IMPORTANT
\begin{spacing}{1} %THESE FOUR
\minitoc % LINES MUST APPEAR IN
\end{spacing} % EVERY
\thesisspacing % CHAPTER
% COPY THEM IN ANY NEW CHAPTER
%%%%%%%%%%%%%%%%%%%%%%%%%%%%%%%%%%%%%%%

This chapter details the technical implementation of the governance bot.
Where Chapter~\ref{ch:method} established the design principles and
high‑level architecture, here we describe how those concepts were
realised in code.  We document the roles of each software module, the
database schema, the algorithms used to tally ballots and the user
interaction flows.  Throughout the chapter we justify design choices
against the literature reviewed in Chapter~\ref{ch:litreview}.  For
example, the interface uses simple buttons and sequential prompts to
minimise cognitive load, in line with usability heuristics, and hides
interim tallies to mitigate the bandwagon effect.  The supported voting
rules are limited to those with polynomial‑time algorithms, reflecting
computational considerations discussed in the literature.

\section{Component Implementation}

Each module of the bot encapsulates a specific responsibility and
communicates with other modules via asynchronous function calls.  The
main components are as follows:

\paragraph{\texttt{main.py} (bot core).}  This file is the entry point
for the Discord client.  It registers command handlers for proposal
submission, voting initiation and administrative functions, and it
schedules background tasks such as \emph{deadline checks} and
\emph{result announcements}.  It maintains an \texttt{asyncio.Queue}
that acts as a task buffer.  When a vote is recorded or a proposal’s
status changes, messages are pushed onto the queue; a worker coroutine
consumes tasks to update tracking embeds or to post final results.  The
bot core also exposes commands for administrators to view and modify
server‑level settings (e.g., default voting mechanism and token
budget).

\begin{algorithm}
    \caption{Queue worker for vote tracking}\label{alg:queue_worker}
    \begin{algorithmic}[1]
        \Procedure{UpdateTrackingWorker}{bot, queue}
            \State Wait until \texttt{bot} is ready
            \While{\texttt{bot} is running}
                \State item $\leftarrow$ dequeue from \texttt{queue}
                \State Extract $guild\_id$ and $proposal\_id$
                \If{either identifier missing}
                    \State Log warning; mark task done; \textbf{continue}
                \EndIf
                \State guild $\leftarrow$ bot lookup of $guild\_id$
                \If{guild exists}
                    \State \Call{UpdateVoteTracking}{guild, $proposal\_id$}
                \Else
                    \State Log warning about missing guild
                \EndIf
                \State Mark task done and sleep for 5~s
            \EndWhile
        \EndProcedure
    \end{algorithmic}
\end{algorithm}

\paragraph{\texttt{proposals.py} (proposal and campaign creation).}
This module defines classes derived from \texttt{discord.ui.Modal} and
\texttt{discord.ui.View} to present interactive forms to users.  When a
user invokes the proposal command, a modal prompts for a title,
description, deadline and set of options.  For weighted campaigns the
modal additionally asks for the number of tokens per voter and the
number of sub‑decisions (e.g., breakfast, transport, activity, dinner).
Upon submission the module validates inputs, writes the proposal and
options to the database and notifies moderators.  It also displays a
view allowing the creator to select a voting mechanism.  The modular
design of these classes means that adding a new field or workflow
involves subclassing and overriding callbacks rather than editing the
bot core.

\paragraph{\texttt{voting.py} (user interaction during voting).}
This module contains a family of \texttt{discord.ui.View} classes that
guide the user through casting a ballot.  For plurality and approval
rules, the view displays a row of buttons corresponding to options and
records the user’s selection or approvals.  For ranked voting
(Borda, instant runoff and Condorcet), the view presents successive
choices asking the voter to select their next most preferred option
until a complete ranking is built.  For weighted voting, the view
displays increment and decrement buttons next to each alternative and a
progress bar indicating how many tokens remain.  When the user
confirms their choices, the view calls into the database to record the
ballot and triggers an update on the task queue.  The design ensures
that all user input is solicited through clear, sequential interactions
despite Discord’s limitations on interface components.  Hiding interim
results prevents social influence and bandwagon effects as discussed in
the literature.

\begin{algorithm}
    \caption{DM voting workflow}\label{alg:dm_voting}
    \begin{algorithmic}[1]
        \Procedure{SendVotingDM}{member, proposal, options}
            \State $proposal\_id \leftarrow$ extract from proposal
            \If{$proposal\_id$ missing} \State \Return \textbf{False} \EndIf
            \State Parse voting mechanism and hyperparameters
            \If{proposal part of a campaign}
                \State Fetch campaign details and enrol voter
                \State Retrieve remaining tokens for member
            \EndIf
            \State Compose embed with proposal metadata, options and deadline
            \If{anonymous voting enabled}
                \State Attach unique identifier for voter
            \EndIf
            \State Instantiate view corresponding to mechanism
            \State Send DM with embed and view
            \State \Return \textbf{True}
        \EndProcedure
    \end{algorithmic}
\end{algorithm}

\paragraph{\texttt{voting\_utils.py} (core logic).}
This module implements the counting algorithms for each supported
mechanism.  It provides pure functions that accept a list of ballots
and, where appropriate, token allocations and return a ranking of
alternatives and the winning option.  Section~\ref{sec:algorithms}
formalises these algorithms.

\paragraph{\texttt{db.py} (persistence layer).}
The database module defines the schema and provides asynchronous
functions to create and query records.  It stores servers, users,
settings, proposals, proposal options, votes and campaigns in separate
tables.  The functions abstract away SQL and return Python objects,
thereby preventing SQL injection and facilitating unit testing.  User
identifiers are hashed before insertion to protect privacy, and
foreign‑key constraints ensure referential integrity.

\paragraph{\texttt{utils.py} and other helpers.}
Auxiliary functions for time formatting, parsing durations, generating
progress bars and performing moderation actions reside in small helper
modules.  These utilities separate peripheral concerns from the core
logic.

\section{Data Model}

The system stores all persistent state in a normalised SQLite
database.  Each table encapsulates a distinct concept, and foreign keys
encode relationships between them.  The main tables are summarised
below:

\begin{description}
    \item[servers] One record per Discord server using the bot.  Fields
    include a unique server identifier (primary key), the server name
    and configuration parameters such as the default voting mechanism,
    whether weighted campaigns are allowed and the token budget.  When
    the bot joins a new server, a corresponding entry is created.
    \item[users] One record per participating user.  The primary key is
    an internal integer identifier; a hashed version of the user’s
    Discord ID is stored instead of the raw identifier to preserve
    anonymity.  Additional columns capture opt‑in status for weighted
    campaigns and optional demographic data if collected via surveys.
    \item[proposals] Each proposal represents a decision that a
    community wishes to make.  Columns include the server ID, the
    proposer’s user ID, a title and description, a deadline timestamp,
    the selected voting mechanism, a status flag (draft, active or
    closed) and whether the proposal is part of a weighted campaign.
    \item[proposal\_options] For each proposal, this table lists the
    alternative options.  Each row stores the option text and a
    foreign key back to the parent proposal.  Composite primary keys
    ensure uniqueness within a proposal.
    \item[votes] Each row records a user’s ballot on a proposal.  For
    ranked and approval votes the ballot is stored as a JSON‑encoded
    array of option identifiers; for weighted voting the tokens
    invested in each option are stored.  Additional columns capture
    the time of submission and whether the voter abstained.  A unique
    constraint on (proposal ID, user ID) prevents duplicate voting.
    \item[campaigns] Weighted campaigns group together multiple
    proposals under a single token budget.  Fields include the
    campaign name, start and end times, the number of tokens per voter
    and the list of proposals involved.  A separate table
    \texttt{user\_campaign\_participation} records how many tokens
    each user has invested so far in a campaign.
\end{description}

Indices are added on foreign keys to accelerate joins.  All database
operations are asynchronous to avoid blocking the event loop.  Hashing
user identifiers and storing only aggregated results complies with
data‑protection regulations and reduces the risk of re‑identification.

\section{Core Algorithms}
\label{sec:algorithms}

This section formalises the voting rules implemented by
\texttt{voting\_utils.py}.  Let $A$ be the set of alternatives, with
\(|A|=n\), and let $V$ be the set of voters.  A ballot is either a
ranking of all alternatives, a set of approved alternatives or a
vector of token allocations.  In the weighted case each voter is
endowed with a fixed budget of tokens $B$; the sum of tokens they
allocate across alternatives must equal $B$.

\begin{algorithm}
    \caption{Tally orchestrator}\label{alg:tally_orchestrator}
    \begin{algorithmic}[1]
        \Procedure{CalculateResults}{$proposal\_id$}
            \State proposal $\leftarrow$ fetch proposal metadata
            \If{proposal missing} \State \Return \textbf{None} \EndIf
            \State votes $\leftarrow$ fetch all votes for proposal
            \State Separate abstentions from effective votes
            \State options $\leftarrow$ load from database or fallback
            \State mechanism $\leftarrow$ determine from proposal
            \State results $\leftarrow$ \Call{count\_votes}{mechanism, effective votes, options, hyperparameters}
            \State Annotate results with abstain counts, token totals and options used
            \State Derive final status based on winner or vote totals
            \State \Return results
        \EndProcedure
    \end{algorithmic}
\end{algorithm}

\paragraph{Plurality.}  Each voter selects a single favourite
alternative.  The tally function counts the number of first‑choice
votes received by each alternative,
\begin{equation}
    s(a) = \sum_{v \in V} \mathbf{1}\{\text{voter }v\text{ ranks }a\text{ first}\},
\end{equation}
where $\mathbf{1}$ is the indicator function.  The alternative(s)
with the largest score win; ties are broken by randomly selecting
among the tied alternatives.  This rule corresponds to the native
Discord poll and suffers from the spoiler effect because only the
top preference is recorded \cite{Black1958}.

\paragraph{Borda count.}  In the Borda rule each ranking is converted
into a point vector.  If voter $v$ ranks alternative $a$ in
position $r_{v}(a)$ (with 1 denoting most preferred), then the
score assigned to $a$ by $v$ is $n - r_{v}(a)$.  The total
score for each alternative is
\begin{equation}
    s(a) = \sum_{v \in V} (n - r_{v}(a)),
\end{equation}
and the alternative with the highest total wins.  The Borda count
captures more information than plurality by considering the entire
ranking and is resistant to some strategic voting, but it violates
independence of irrelevant alternatives \cite{Borda1784}.

\paragraph{Approval voting.}  Each voter approves of any number of
alternatives.  The tally function counts the number of approvals per
alternative,
\begin{equation}
    s(a) = \sum_{v \in V} \mathbf{1}\{a \in A_{v}^{\text{approved}}\},
\end{equation}
where $A_{v}^{\text{approved}}$ is the set of alternatives
approved by voter $v$.  The alternative(s) with the highest approval
score win.  Approval voting reduces vote splitting and allows voters
to express indifference between multiple acceptable options
\cite{Black1958}.

\paragraph{Instant runoff voting (IRV).}  Also known as ranked
choice voting, IRV simulates a series of runoff elections.  At each
round the alternative with the fewest first‑choice votes is eliminated
and its votes are transferred to the next preferred remaining
alternative on each ballot.  The process continues until a single
alternative remains.  Formally, let $A^{(k)}$ be the set of
candidates remaining in round $k$.  Compute scores
\(s^{(k)}(a) = \sum_{v \in V} \mathbf{1}\{\text{first choice of }v\text{ in }A^{(k)}\text{ is }a\}\).  Eliminate the alternative with the smallest
$s^{(k)}$; set $A^{(k+1)} = A^{(k)} \setminus \{\text{eliminated}\}$ and repeat.  IRV guarantees a majority winner if one exists and avoids
the no–show paradox but is vulnerable to non‑monotonicity
\cite{Bartholdi1989}.

\paragraph{Condorcet method.}  A Condorcet winner is an alternative
that defeats every other alternative in head‑to‑head majority
comparisons.  Construct the pairwise majority matrix $M$ where
$M_{a,b}$ is the number of voters who prefer $a$ over $b$.
If there exists an alternative $c$ such that $M_{c,b} > M_{b,c}$
for all $b \neq c$, then $c$ is the Condorcet winner and is
selected.  If no such alternative exists (i.e., the majority
preferences form a cycle), a Condorcet‑consistent method such as the
Schulze algorithm can be applied to compute a ranking.  Condorcet
methods respect the majority criterion and are highly expressive but
may not always produce a unique winner \cite{Condorcet1785}.

\paragraph{Weighted (token) voting.}  To capture intensity of
preferences, each voter is given a budget $B$ of tokens to distribute
among the alternatives.  Let $t_{v,a}$ denote the number of tokens
invested by voter $v$ in alternative $a$, with the constraint
$\sum_{a \in A} t_{v,a} = B$.  The weighted score of each
alternative is
\begin{equation}
    s(a) = \sum_{v \in V} t_{v,a}.
\end{equation}
The alternative with the highest token sum wins.  Weighted voting
generalises approval voting (each approval contributes one token) and
is related to quadratic voting in which the cost of casting multiple
votes increases quadratically with the number of votes.  Our design
fixes a linear cost in tokens to simplify the user experience and
allocates an equal budget to each voter to preserve fairness.

In all rules ties are broken by either drawing at random among tied
alternatives or by applying a deterministic tie‑breaking rule (e.g.,
lexicographic order).  When run within a weighted campaign, the raw
vote counts and token sums are also recorded for post‑hoc analysis of
participation and investment behaviour.

\section{UI and Interaction Flow}

User interactions with the governance bot unfold along a small number of
well‑defined pathways: proposal submission, campaign configuration,
voting and result announcement.  Designing these pathways required
adapting complex social choice tasks to the limited set of UI
primitives provided by Discord.  Messages may contain up to five
buttons, and modal dialogs support only short text inputs; there are no
native drop‑down menus or drag‑and‑drop widgets.  The bot therefore
implements a series of sequential views that guide the user through each
step while enforcing input validation on the client side.  Hiding
interim tallies until voting closes is a deliberate choice to mitigate
bandwagon effects and preserve independence of irrelevant alternatives
\cite{Arrow1951}.

\paragraph{Proposal and campaign creation.}  A user initiates a decision
by invoking a slash command (e.g., \texttt{/proposal}) in the proposals
channel.  The bot presents a modal requesting a title, a detailed
description, a closing deadline and a list of options.  For weighted
campaigns the modal also asks for the number of tokens per voter and the
number of sub‑decisions (e.g., breakfast, transport, activity, dinner).
Upon submission the bot validates the inputs (ensuring that the deadline
is in the future and that options are non‑empty), stores the proposal
and its options in the database and notifies moderators to approve or
reject it.  Once approved, the bot posts an embed summarising the
proposal and provides a button for participants to opt in.

\paragraph{Voting interaction.}  When a proposal becomes active, the bot
sends each participating user a direct message containing a view tailored
to the selected voting mechanism.  For plurality and approval voting,
users see a row of buttons corresponding to alternatives and may select
one or toggle approvals.  For ranked rules, the view repeatedly
prompts the voter to choose their next favourite alternative until a
complete ranking is obtained.  The UI explicitly prevents
duplicate selections and displays the current partial ranking.

\paragraph{Weighted voting and token allocation.}  In weighted
campaigns each voter receives a budget of tokens (set by the campaign initiator)  The voting view displays increment and decrement
buttons next to each alternative along with a progress bar showing the
remaining budget.  Users may adjust their allocations until they are
satisfied and then confirm their investment.  Once confirmed the
allocation is recorded and cannot be altered.  This procedure is
described in detail in Appendix~\ref{appendix:token}.  The sequential
allocation interface is intentionally simple to minimise cognitive load
and align with user‑experience guidelines.

\paragraph{Result publication and logging.}  After the deadline
elapses the bot calls the appropriate tally function from
\texttt{voting\_utils.py} and posts an embed summarising the outcome.
For transparency the embed can include a table of raw vote counts or
token sums (if the community opts in), but interim results are never
displayed.  All data required for computing participation rate, time to
vote and outcome variance are logged for subsequent analysis.  These
measures are linked to the metrics defined in Chapter~\ref{ch:experiments}.