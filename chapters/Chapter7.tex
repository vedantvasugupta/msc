\doublespacing % Do not change - required

\chapter{Reflections}
\label{chLast}

%%%%%%%%%%%%%%%%%%%%%%%%%%%%%%%%%%%%%%%
% IMPORTANT
\begin{spacing}{1} %THESE FOUR
\minitoc % LINES MUST APPEAR IN
\end{spacing} % EVERY
\thesisspacing % CHAPTER
% COPY THEM IN ANY NEW CHAPTER
%%%%%%%%%%%%%%%%%%%%%%%%%%%%%%%%%%%%%%%

% This chapter has been migrated from the original LastChapter and
% renumbered.  It provides a reflective assessment of the project from
% legal, ethical, environmental and inclusion perspectives, discusses
% quality management practices and introduces a new section on project
% limitations and technical challenges.

This final substantive chapter provides a reflective assessment of the
project from legal, ethical, environmental and inclusivity perspectives.
It also discusses quality management practices employed during
development.  Such reflections are required by the programme’s learning
outcomes and ensure that engineering work is situated within its broader
social context.

\section{Legal and Ethical Matters}

\textbf{Legal considerations}.  The governance bot processes user input on
Discord servers.  Although the bot stores only minimal metadata—such as
anonymised votes and tokens—it nevertheless interacts with personal data.
The implementation adheres to the spirit of data protection regulations
such as the UK GDPR.  No personally identifiable information is stored;
user identifiers are hashed before being persisted, and only aggregate
statistics are published.  Users are informed about how their data will
be used, and participation in weighted campaigns is strictly opt‑in.
Intellectual property rights were respected: external libraries were used
under appropriate licences, and contributions to the open‑source
repository are attributed.

The system implements privacy by design.  Each user’s unique Discord identifier
is passed through a cryptographic one‑way hash function before being
stored, and the original identifier is never written to the database.
The bot stores token allocations only in aggregated form; individual
allocations are deleted once the result has been announced.
Participation in weighted campaigns is explicitly opt‑in and users can
withdraw their consent at any time, upon which their votes and token
allocations are removed from the data store.  These practices align
with data‑protection principles of minimisation, purpose limitation
and the right to be forgotten.

\textbf{Ethical considerations}.  Algorithmic governance systems can
embed biases that disadvantage certain users.  To mitigate this, the bot
offers multiple voting mechanisms and allows communities to choose their
preferred method.  Weighted voting raises ethical questions about
commodifying preferences.  Tokens were framed as a budget for expressing
intensity rather than as a currency; no real financial stakes are
involved.  The experimental protocol received informal ethical approval
from the supervisor, and participants provided informed consent.  The
results were anonymised, and participants could withdraw at any time.

When designing the token system we avoided analogies that could be
interpreted as monetising influence.  Tokens are allocated equally to
all participants and cannot be purchased, transferred or saved across
decisions.  The budget of one hundred tokens provides sufficient
granularity to express intensity without overwhelming users; future
studies could experiment with different budgets.  Instructions
emphasise that tokens signify the strength of preference rather than
financial value, and the UI prevents over‑investment by
enforcing the budget constraint.  These choices are intended to
preserve the normative distinction between civic participation and
market exchange.

Beyond compliance with data protection statutes, algorithmic governance
raises broader questions of legitimacy and power.  The
Engage–Envision–Enact framework reminds us that democracy should not
be assumed to be a universal end state and that legitimacy derives from
meaningful, informed and revocable participation.  Designers must
therefore ensure that participants understand the voting mechanisms and
have the ability to contest outcomes or propose alternatives.  The
socially‑guided machine learning literature warns of the dangers of
``techno‑feudalism'' if AI systems mediate decisions without proper
oversight \cite{Mertzani2025SGML}.  While our bot does not employ
machine learning, future extensions (such as the proposed
\emph{what‑if} predictor) must be scrutinised for potential biases and
their impact on power relations within the community.

\section{Environmental and Social Impact}

\textbf{Environmental impact}.  The bot runs on modest cloud hardware
and uses lightweight data storage.  Energy consumption is minimal
compared with compute‑heavy AI systems.  The project’s main environmental
contribution lies in its potential to support collaborative decision‑making
that could, for example, promote sustainable activities (e.g., choosing a
low‑carbon outing) or facilitate coordination of community projects.  The
source code and documentation are released under an open‑source licence,
enabling reuse and reducing duplication of effort.

\textbf{Social impact}.  By empowering communities to participate in
governance, the bot can foster a sense of ownership and collective
responsibility.  However, digital divides persist: not all community
members have equal access to or familiarity with Discord.  To mitigate
exclusion, future versions could integrate with alternative platforms or
provide offline participation modes.  There is also a risk that weighted
voting may privilege vocal or well‑resourced participants if not carefully
explained.  Transparent rules and community moderation are essential to
prevent misuse.

Digital divides manifest along multiple dimensions: access to devices and
internet connectivity, familiarity with online platforms and ability to
navigate UIs.  To address these issues, future iterations of
the governance tool should implement cross‑platform clients—for example,
a web portal accessible via browsers, SMS‑based interfaces for users
without smartphones or dedicated mobile applications with offline
support.  Providing multilingual options and clear tutorials will help
broaden participation.  The token system must be accompanied by
transparent explanations and community guidelines to avoid perceptions
that influence is being bought.  Community moderation and participatory
rule‑setting processes can help ensure that weighting schemes are
understood and accepted.

Self‑organising decision systems must also guard against Goodhart’s
law: optimising for a single metric (for example, participation rate
or energy use) can distort behaviour and undermine the ultimate goals of
fairness or sustainability.  Implementing multiple measures and
qualitative feedback channels can help maintain balanced evaluation
criteria.

\section{Equality, Diversity and Inclusion}

The design sought to be inclusive of users with diverse abilities and
backgrounds.  Buttons and modal interfaces were labelled clearly, and
instructions avoided jargon.  Surveys asked participants about ease of
use to identify accessibility issues.  Weighted voting allows
individuals to distribute preference intensity, which can benefit
minorities by signalling strong support for less popular options.
Nonetheless, weighted schemes could disadvantage users unfamiliar with
token allocations.  Future work should include accessibility testing
with users from varied age groups, technical proficiencies and
disabilities.  Features such as screen‑reader compatibility, high‑contrast
colour schemes, keyboard navigation, descriptive alternative text for
icons and clear error prompts are essential to ensure that the bot is
usable by people with visual or motor impairments.  Providing input
methods beyond buttons—such as slash commands or voice interfaces—can
accommodate different interaction preferences.

Ensuring inclusivity also involves linguistic and cultural
considerations.  Many online communities span multiple countries and
language groups; therefore, future versions of the bot should support
multilingual interfaces and consider cultural differences in voting
norms.  Integration with platforms beyond Discord—including web portals
that work with screen readers, SMS‑based interfaces for users without
smartphones and printed ballots for in‑person meetings—could broaden
participation and bridge digital divides.  Engagement with community
stakeholders from the earliest design stages will help surface diverse
needs and avoid assumptions about homogeneity.  Co‑design workshops and
iterative user testing can ensure that the governance tool remains
accessible, equitable and culturally sensitive.

\section{Quality Management Systems}

Quality management was integral throughout the project.  The GitHub
repository used version control to track changes, with meaningful commit
messages explaining the rationale for modifications.  Continuous
integration scripts automatically ran unit tests on key functions, such as
proposal creation and vote counting, to detect regressions.  Code was
organised into modules with clear interfaces to facilitate peer review and
future maintenance.  Documentation was maintained in README files and
inline comments.  During deployment, logging and error handling were
implemented to monitor runtime behaviour and diagnose issues quickly.

Regular peer review sessions were held to inspect code for clarity,
conformance to style guidelines and potential security issues.  Issues
were tracked using GitHub’s issue tracker, and each feature branch
underwent automated testing and manual review before merging.  The use
of continuous integration ensured that refactoring or adding new voting
rules did not break existing functionality.  These practices align with
industry standards and facilitate future maintenance and collaboration.

\section{Project Limitations and Technical Challenges}

The design and implementation of the governance bot were subject to
several practical limitations and technical challenges.  First, the
Discord platform provides only a handful of interaction widgets: at most
five buttons per message, short text inputs in modals and no native
drag‑and‑drop or dropdown selectors.  Implementing complex voting rules
within these constraints required creative solutions.  For instance,
ranking alternatives for the Borda, IRV and Condorcet methods was
realised through a sequence of prompts asking users to select their next
preferred alternative rather than through a single ranked form.  Token
allocation in weighted voting involved increment and decrement buttons and
a progress bar to convey the remaining budget.  These sequential
interactions increase the number of clicks compared with native polls and
may introduce fatigue; however, they were necessary to comply with the
platform’s API limitations.

Second, the absence of real-time feedback mechanisms on Discord limited
the ability to provide immediate visualisations of current vote tallies.
To mitigate bandwagon effects—a phenomenon where voters are swayed by
interim results—the bot deliberately suppresses interim counts until
voting closes.  This design choice is grounded in the literature on the
bandwagon effect and aims to preserve independence of irrelevant
alternatives.  Nevertheless, it means that participants do not receive
dynamic feedback on how their community is voting, which may reduce
engagement for users accustomed to live polls.

Third, computational efficiency informed the choice of voting rules.  Some
systems in social choice, such as Kemeny’s method, are NP-hard to compute
exactly.  Our implementation therefore supports only polynomial-time
algorithms such as plurality, Borda, approval, instant runoff and
Condorcet (with a simple majority matrix).  While this decision aligns
with feasibility constraints, it excludes methods that satisfy stronger
fairness axioms but incur intractable computation.  Future work could
explore approximate algorithms or off-chain computation to expand the
available mechanisms.

Finally, developing a robust and secure bot requires careful handling of
concurrency and data persistence.  The event-driven nature of
\texttt{discord.py} demands asynchronous programming; mismanaging
coroutines can lead to deadlocks or lost updates.  Similarly, ensuring
that the database is updated atomically and that race conditions do not
corrupt state was a key challenge.  Logging, extensive testing and
incremental deployment helped address these issues but required
substantial engineering effort.  Recognising these limitations and
challenges provides valuable lessons for future iterations of the
governance platform.