\doublespacing % Do not change - required

% The \conclusions macro in the main template creates the chapter heading.
% Do not include a \chapter command here.
\chapter{Conclusions}
\label{ch:conclusions}

This chapter summarises the contributions of the thesis and outlines
opportunities for future research and development.  In keeping with the
requirements of the programme, the conclusions provide a critical
evaluation of the project’s successes and limitations.

\section*{Conclusions}

This thesis presented the design and implementation of a participatory
governance bot for Discord.  Building on principles from social choice
theory, PD and algorithmic governance, the bot supports
multiple voting mechanisms and a weighted campaign mode that allows users
to express the intensity of their preferences.  A detailed methodology
was developed and implemented to evaluate the bot through a ``Day~Out'' scenario
with pre‑ and post‑voting surveys and objective metrics.  The study recruited
ten participants who completed both a plurality vote and a weighted
token‑allocation vote.  The empirical results showed that the weighted mechanism
improved perceptions of fairness and satisfaction on all measured items and was
overwhelmingly preferred for future decisions.  Only one component of the
``Day~Out'' plan differed between the two methods—the afternoon activity—indicating
that cardinal mechanisms can influence outcomes when intensity of preference is
unequally distributed without radically altering consensus plans.  The modular
architecture, developed using \texttt{discord.py}, ensures extensibility
and separates concerns across database management, proposal creation and
vote counting.  Throughout the project, legal and ethical obligations,
environmental and social impacts, inclusivity and quality management were
considered.

The work illustrates how computational social choice can be operationalised
in a popular communication platform.  It provides a proof‑of‑concept tool
that communities can adapt to their own governance needs, aligning with
the engage–envision–enact philosophy of self‑organised socio‑technical
systems.  By empowering users to select and modify voting rules, the bot
contributes to a broader agenda of value‑sensitive and power‑sensitive
design.

\section*{Future Work}

Several avenues remain for further research and development:

\begin{itemize}
    \item \textbf{Larger and more diverse studies}.  Future experiments
    should involve more participants from varied backgrounds and explore
    high‑stakes decision scenarios.  Comparative studies with existing
    governance tools would contextualise usability and effectiveness.
    \item \textbf{Enhanced UI}.  Graphical token sliders,
    adaptive tutorials and accessibility features (e.g., screen‑reader
    support) could reduce cognitive load and broaden participation.  As
    Discord introduces new UI components such as dropdowns for slash
    commands, the bot could integrate these elements to support ranked
    choice voting more naturally.
    \item \textbf{Adaptive parameters}.  The current implementation uses
    static thresholds (e.g., quorum sizes and token budgets).  Machine
    learning could be employed to tune these parameters based on past
    participation patterns, user preferences and fairness metrics.
    \item \textbf{Cross‑platform integration}.  Extending the system to
    other communication platforms (e.g., Slack or Matrix) and supporting
    offline participation modes would enhance inclusivity and resilience.
\end{itemize}

\section*{Hybrid Human-AI Governance and 'What-If' Predictors}

A significant opportunity for future work lies in extending the bot into a
platform for hybrid human-AI governance.  The current system provides a
crucial foundation for this line of research by capturing structured
quantitative data on community preferences (votes, rankings, and token
allocations).  This data can fuel a machine-learning-based
\emph{what-if} predictor to help communities deliberate more effectively.

Such a tool would move beyond simple outcome prediction.  By integrating
Large Language Models (LLMs), the system could analyze the qualitative
discourse from the debates that precede a vote.  While the current bot
captures the \emph{what} of a decision, a hybrid approach would also
process the \emph{why}, by understanding the arguments, sentiments, and
counterarguments raised in chat discussions.  This allows for a richer
form of analysis where the system could help communities explore
counterfactual scenarios.  For example, users could ask: ``What might the
outcome be if we used Borda count instead of plurality, given the
second-preference arguments made in the discussion?'' or ``How might
sentiment change if we rephrased the proposal to address the primary
concerns raised by its opponents?''.

This hybrid model—fusing quantitative voting data with qualitative
insights from deliberation—could serve as a powerful deliberative aid.
However, it must be designed with caution.  As researchers have noted,
such tools risk shifting power away from users if not implemented with
transparency and user control at the forefront.  The AI should not be an
oracle that dictates outcomes, but a tool that helps humans understand
the complex trade-offs of collective decision-making.  Its purpose is to
augment, not automate, human judgment, thereby avoiding the pitfalls of
``techno‑feudalism'' while advancing the potential for more informed and
reflective democratic processes.

In conclusion, the project demonstrates the potential viability of
participatory decision‑making tools in contemporary digital environments.
By bridging theoretical insights from social choice and socio‑technical
systems with practical implementation, it lays a foundation for
refining the approach and evaluating it at larger scales, advancing both
the science and the practice of collective governance.